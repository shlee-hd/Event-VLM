\documentclass[runningheads]{llncs}

% TODO FINAL: Comment out review mode for camera-ready.
\usepackage[review,year=2026,ID=*****]{eccv}
%\usepackage{eccv}

\usepackage{eccvabbrv}
\usepackage{graphicx}
\usepackage{booktabs}
\usepackage{amsmath}
\usepackage{amssymb}
\usepackage[accsupp]{axessibility}

% TODO FINAL: switch to non-review hyperref option.
\usepackage[pagebackref,breaklinks,colorlinks,citecolor=eccvblue]{hyperref}
%\usepackage{hyperref}

\begin{document}

\title{CERA: Causal Event Reasoning and Attribution for Real-Time Surveillance}
\titlerunning{CERA}

\author{Anonymous Authors}
\authorrunning{Anonymous Authors}
\institute{Anonymous Institution\\
\email{anonymous@eccv2026.org}}

\maketitle

\begin{abstract}
Real-time surveillance systems need more than anomaly flags or descriptive captions.
Operational response requires causal accounts that connect events, agents, and outcomes under strict latency constraints.
We introduce \textbf{CERA} (\textbf{C}ausal \textbf{E}vent \textbf{R}easoning and \textbf{A}ttribution), a framework direction for online surveillance analysis that prioritizes three requirements: causal fidelity, evidential grounding, and runtime efficiency.
This draft establishes the problem framing and design requirements as the foundation for method and evaluation development.
\keywords{Causal Event Reasoning, Attribution, Surveillance, Efficient Inference}
\end{abstract}

\section{Introduction}

\paragraph{Why Causality Matters in Surveillance.}
Modern surveillance models can detect anomalies or generate captions, but practical response requires more: a causal account of what happened, what triggered it, and which entities were responsible.
In safety-critical settings, this distinction affects intervention priority, accountability, and prevention planning.

\paragraph{Problem Setting.}
We consider online surveillance streams where decisions must be both timely and explainable.
For each event segment, the system should produce (1) an event statement, (2) an attribution statement linking causes to outcomes, and (3) supporting evidence references.
We refer to this objective as \textit{causal event reasoning and attribution}.

\paragraph{Current Gap.}
Existing pipelines are often optimized for a single objective.
Detection-focused systems emphasize recall but provide limited causal structure.
Caption-focused systems can describe scenes fluently but often blur causality and agency.
Offline reasoning approaches are expressive but difficult to deploy with strict real-time constraints.

\paragraph{CERA.}
We introduce \textbf{CERA}, a framework direction for real-time causal event reasoning and attribution.
CERA is organized around three constraints: \textit{causal fidelity} (correct cause-effect structure), \textit{evidential grounding} (traceable support in observed frames), and \textit{runtime efficiency} (practical throughput for continuous streams).
This paper starts by fixing these requirements as first-class design targets.

\paragraph{Scope of This Draft.}
This manuscript is built from a clean-room start.
The current version focuses on defining the CERA problem statement and system requirements before expanding to full method specification and large-scale experiments.

\paragraph{Contributions.}
\begin{itemize}
    \item We define a real-time surveillance task for causal event reasoning and attribution with explicit outputs for event, cause, and evidence.
    \item We propose CERA as a framework direction centered on causal fidelity, evidential grounding, and runtime efficiency.
    \item We establish a starting evaluation perspective that jointly considers causal quality and deployment-time performance.
\end{itemize}

\section{Method}
\textbf{TODO.} Specify CERA components, notation, assumptions, and computational trade-offs.

\section{Experiments}
\textbf{TODO.} Define datasets, protocol, baselines, metrics, and statistical plan.

\section{Conclusion}
\textbf{TODO.} Summarize findings and next validation steps.

\bibliographystyle{splncs04}
\bibliography{main}

\end{document}
