\documentclass[runningheads]{llncs}

% TODO FINAL: Comment out review mode for camera-ready.
\usepackage[review,year=2026,ID=*****]{eccv}
%\usepackage{eccv}

\usepackage{eccvabbrv}
\usepackage{graphicx}
\usepackage{booktabs}
\usepackage{amsmath}
\usepackage{amssymb}
\usepackage[accsupp]{axessibility}

% TODO FINAL: switch to non-review hyperref option.
\usepackage[pagebackref,breaklinks,colorlinks,citecolor=eccvblue]{hyperref}
%\usepackage{hyperref}

\begin{document}

\title{CERA: Causal Event Reasoning and Attribution for Real-time Surveillance}
\titlerunning{CERA}

\author{Anonymous Authors}
\authorrunning{Anonymous Authors}
\institute{Anonymous Institution\\
\email{anonymous@eccv2026.org}}

\maketitle

\begin{abstract}
This clean draft starts from the core keyword \textbf{CERA} (\textbf{C}ausal \textbf{E}vent \textbf{R}easoning and \textbf{A}ttribution).
We build the paper from first principles and add technical content incrementally.
\keywords{Causal Event Reasoning, Attribution, Surveillance, Efficient Inference}
\end{abstract}

\section{Introduction}
\textbf{TODO.} Define problem, deployment setting, and why causal event reasoning is needed in real-time surveillance.

\section{Method}
\textbf{TODO.} Specify CERA components, notation, assumptions, and computational trade-offs.

\section{Experiments}
\textbf{TODO.} Define datasets, protocol, baselines, metrics, and statistical plan.

\section{Conclusion}
\textbf{TODO.} Summarize findings and next validation steps.

\bibliographystyle{splncs04}
\bibliography{main}

\end{document}
